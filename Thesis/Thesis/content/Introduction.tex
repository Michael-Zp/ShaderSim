 %!TEX root = ../dokumentation.tex

\chapter{Introduction}\label{cha:Introduction}

\paragraph{Explanation of debugging}
\label{paragraph:debuging}

"Debugging is the process of locating and removing faults in computer programs" according to \myCite{Collins_debugging.2014}. The steps that are part of the debugging process are reproducing the problem, identifying the source of the problem and fixing the problem. All of these steps can be done manually but there are ways to improve and accelerate this process.

To find problems there is the option of writing automated tests, inserting debug outputs on the console into the source code or writing states into log files. This enables the programmer to find anomalies before, while and after running the program.

When a way is found to reproduce the problem, to find the source of it the manual way is to increase the amount of debug outputs around the problematic part of the code and confine the point in the code at which the error occurs.

"Everyone knows that debugging is twice as hard as writing a program in the first place. So if you're as clever as you can be when you write it, how will you ever debug it?" according to \myCite{Kernighan.1982}. As this states, debugging is a quite exhausting and time consuming task. For that reason for most programming languages there are tools to aid the programmer to narrow down the source of the bug with following methods:

\begin{itemize}

\item Enabling the user to set breakpoints at which the program pauses and he can inspect the values of the variables directly within the code. By continuing the program to move to the next breakpoint or by going forward through the code step by step the point where the error occurs can be found. It is also possible to be able to add conditions to the breakpoints describing a state that has to be fulfilled for the debugger to pause. \myCite{Undo.2019}


\item Have the code throw an exception when unwanted behavior occurs and stop at this exception. By saving a stack of the calls which occurred before the exception was thrown or dumping the buffer, the programmer can retrace where the error may be found.\myCite{Jetbrains.2019}


\item Reverse debugging records all program activities and thereby it is possible to move backwards in addition to forward stepping from a set breakpoint and see the changes in the variables and the calls in the code leading to the problem.\myCite{Undo.2019}

\end{itemize}

When the source of the problem is found the final step of fixing the problem is to correct the code.

\paragraph{Structure of the graphics pipeline}
\label{paragraph:pipeline}

The graphics pipeline also known as rendering pipeline differentiates from framework to framework but the basic structure is identical.\myCite{Khronos_pipeline.2019} \myCite{Microsoft_pipeline.2018}

In the first stage the vertices are specified. The vertices are set up in an ordered list. By setting a primitive type it is determined how these vertices define the boundaries of primitives. Attributes are linked to the vertices adding data to them.

The second stage is the execution of the vertex shader for each defined vertex with its data. For each vertex the vertex shader generates an output vertex with data.

After the vertex stage multiple, optional stages can occur. Examples for these are the Tesselation stage or the Geometry stage where additional shaders are executed to implement additional changes to the vertices or even remove or add vertices.

In the following stage there are multiple optimization steps, the primitive assembly and the rasterization. The perspective divide and the viewport transformation are calculated. Through clipping, which is optional, primitives that overlap the edge of the viewing volume are split into multiple primitives and primitives outside the viewing volume are removed. Primitives are assembled according to the primitive type, so a list of primitives is constructed out of the list of vertices where each primitive is constructed out of multiple vertices. Face culling can occur as additional optimization removing primitives that face certain directions in the window space. The primitives are then rasterized into fragments within the raster of the render result. The data values linked to the vertices are automatically interpolated within the primitive and added to the corresponding fragments.

Within the following fragment stage the fragment shader will be executed for each single fragment generating the outputs. There is always a depth buffer and a stencil buffer as an output. Usually, a color output is also defined within the vertex shader, but it is possible to execute the pipeline without setting a fragment shader. Then a default fragment shader runs, where only depth and stencil buffers are set.

After the fragment stage the output of the render process is calculated. For each sample there are multiple optional filters that can be activated to change which fragments are used within this stage. Examples for these filters are the scissor test where a fragment is ignored when it's pixel lies outside of the rectangle of the screen or the depth test where the fragments order is changed when its depth fulfils specific, user defined conditions compared to another fragment's depth within the same sample. After all active filters were executed the color blending happens where the color between multiple fragments within the same sample is calculated together with a specific blending operation. The final data is written to the framebuffer.

The two most known uses of the graphics pipeline are within the graphics libraries OpenGL and Direct3D as shown in  \autoref{fig:pipeline_opengl} and \autoref{fig:pipeline_direct3d}

\begin{figure}[h!]
  \centering 
  \includegraphics[scale=0.3]{Pipeline_OpenGL.png}
  \caption[Graphics pipeline by OpenGL \myCite{Khronos_pipeline.2019}]{Graphics pipeline by OpenGL}
  \label{fig:pipeline_opengl}
\end{figure}

\begin{figure}[h!]
  \centering 
  \includegraphics[scale=0.4]{Pipeline_Direct3D.png}
  \caption[Graphics pipeline by Direct3D \myCite{Microsoft_pipeline.2018}]{Graphics pipeline by Direct3D}
  \label{fig:pipeline_direct3d}
\end{figure}

\paragraph{Problem with debugging shaders in the graphics pipeline}
\label{section:problems}

A shader is a program running on the GPU thereby mostly running as part of the graphics pipeline.\myCite{Khronos_shader.2019} The exception for this behavior is the compute shader which is independent from the graphics pipeline.\myCite{Khronos_compute.2019}

There are ways to debug shaders on the graphics pipeline, as shown in \autoref{section:debuggingMethods}, but there are no general solutions for aiding the user in this debugging process.\myCite{Ciardi.2015}

While most CPUs are very broad in their functionality and support debugging by itself a GPU is more specialized in the way it functions and usually does not have the option to pause the code to enable inspections at runtime and does not even have access to a console or logger to write states into.\myCite{Fox.2017} As explained in more detail in \autoref{section:debuggingMethods_drivers} it is possible to enable debugging on the GPU with specialized drivers for specific hardware.\myCite{Nvidia_debug.2019}\myCite{Microsoft_debug.2016}

\paragraph{Existing approaches for compute shaders}

There are approaches which enable debugging of compute shader code by translating it from another language. The code is written in a language for the CPU which supports debugging with tools. It is run and debugged on the CPU while it is translated to the shader language to run as a compute shader on the GPU.
See more in \autoref{section:computeApproaches}.

\paragraph{Objective of creating a general solution for debugging shaders in the graphics pipeline}
\label{paragraph:objective}

The objective of this work is to create a general solution to enable debugging of shaders within the graphics pipeline. The goals this solution should fulfill are the following:

\begin{itemize}
	\item Different methods to assist the programmer in debugging as shown in \autoref{paragraph:debuging} are usable.
	\item The solution is not dependent on the use of specific graphics cards or drivers.
	\item It is possible to switch between a mode where debugging is enabled and a mode where the shaders run as usual, so the program can run with the full performance and without interference of the debugger.
	\item The resulting output per render iteration of the debugger is close to the output of the program with the undebugged shader. It is close enough that the programmer can see what the rendering result without the debugger would look like. Errors like those resulting from the use of float variables with their inaccuracies are tolerable because there are tolerances within human perception where minimal changes in position or color within a rendered result do not matter.\myCite{Franz.2006}
	\item Performance is not a major requirement while running the debugger. It is possible to see the output and the values of the shader within each frame and iterate through the frames. It is not necessary to view the result in the speed of the final application while debugging. The programmer uses time to inspect the values within the shader which would not be possible with changing variables at high speeds like the 60 frames per second a usual graphics application has.\myCite{Christensson.2015} The debug application should fulfil acceptable response times for user interfaces. "10 seconds is about the limit for keeping the user's attention focused on the dialogue." according to \myCite{Nielson.1993}.
\end{itemize}




